\documentclass[12pt,a4paper]{scrartcl} 
\usepackage[utf8]{inputenc}
\usepackage[english,russian]{babel}
\usepackage[colorlinks,urlcolor=blue]{hyperref}
\begin{document}
\thispagestyle{empty}
\begin{center}
Министерство образования и науки Украины
\end{center}
\begin{center}
Одесский национальный университет им. И.И.Мечникова
\end{center}
\par\bigskip
\par\bigskip
\par\bigskip
\par\bigskip
\par\bigskip
\par\bigskip
\par\bigskip
\par\bigskip
\par\bigskip
\par\bigskip
\par\bigskip
\par\bigskip
\par\bigskip
\par\bigskip
\par\bigskip
\begin{center}
\Large\textbf{Отчет по учебной практике}
\end{center}
\par\bigskip
\par\bigskip
\par\bigskip
\par\bigskip
\par\bigskip
\par\bigskip
\par\bigskip
\par\bigskip
\begin{flushright}
Выполнил:\\
Студент III курса\\
Специальности «Прикладная математика»\\
Чеповский Иван\\
Преподаватель:\\
Огуленко А.П.
\end{flushright}
\par\bigskip
\par\bigskip
\par\bigskip
\par\bigskip
\par\bigskip
\par\bigskip
\par\bigskip
\par\bigskip
\par\bigskip
\par\bigskip
\begin{center}
г. Одесса, 2016
\end{center}
\newpage
\begin{center}
\Large\textbf{Постановка задачи}
\end{center}
\begin{enumerate}
\item Знакомство с работой Git
\item Знакомство с Python
\item Проект на Python
\item Освоение LaTeX
\end{enumerate}
\newpage
\begin{center}
\Large\textbf{Ход работы}
\end{center}
\begin{enumerate}
\item В ходе знакомства с работой Git я научился создавать свой репозиторий, удаленно работать с ним, а также освоил его главную функцию – возможность командной разработки приложения.
\item В ходе знакомства с Python я получил базовые навыки работы с данным языком программирования. Было выполнено 8 заданий, связанных с работой циклов, вводом и выводом данных, работой с файлами и различными библиотеками.
\item После знакомства с языком программирования Python, используя полученные навыки, нужно было создать свое приложение, работая в команде. Я работал в команде из 4 человек. Нами было разработано приложение по расчету совместимости двух людей.
\item На освоение LaTeX было отведено 3 задания. Первое из них – это набор текста из данного doc-файла, второе – набор текста из данной рукописи и третье – используя пакет Beamer, создать презентацию своей курсовой. А также данный отчет был написан, используя LaTeX.
\end{enumerate}
\par\bigskip
\begin{center}
\Large\textbf{Выводы}
\end{center}
\parbox{14,5 cm}{\parindent=1 cm В ходе практики мною было освоено:}
\begin{enumerate}
\item Git
\item Python
\item LaTeX
\item Beamer
\end{enumerate}
\parbox{14,5 cm}{\parindent=1 cm Данная учебная практика была довольно полезной для меня, так как работа с Git – это основа при командной разработке любого приложения, язык программирования Python в последнее время набирает все большую популярность, а с помощью LaTeX и Beamer любой математик должен уметь верстать свои научные труды и презентации. }
\end{document}